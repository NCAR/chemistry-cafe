\documentclass[titlepage]{article}

\usepackage{amsmath}
\usepackage{chemformula}
\usepackage{fancyvrb}
\usepackage{hyperref}
\usepackage{lipsum}
\usepackage[margin=1in]{geometry}
\usepackage{xcolor}

\DefineVerbatimEnvironment{blockcode}
  {Verbatim}
  {fontsize=\small,formatcom=\color{blue}}

\begin{document}

\title{TS1 Custom Rate Constants}
\author{Matt Dawson}
\maketitle

%%%%%%%%%%%%%%%%%%%%%%%%%%%%%%%%%%%%%%%%%%%%%%%

\section{Introduction}

To output chemical mechanisms from the Chemistry Caf\'{e} in Music Box format, the custom rate constants in CAM need to be addressed individually and either rearranged into one or more standard Music Box reaction types, or used to create new reaction types in Music Box. For now aerosol surface reactions will be ignored. For the gas-phase reactions of the TS1 mechanism, it appears that all but one reaction that uses a custom rate constant can be rearranged into sets of standard Music Box reactions. The one remaining reaction has been added to Music Box as the `ternary chemical activation' reaction type.

The following sections present the original rate constant functions for all non-standard gas-phase TS1 reactions from \verb>cam/src/chemistry/mozart/mo_usrrxt.F90>, include notes I made as I navigated this code for the first time, and present a proposed refactoring into standard Music Box reaction types.


%%%%%%%%%%%%%%%%%%%%%%%%%%%%%%%%%%%%%%%%%%%%%%%

\section{usr\_DMS\_OH}

\begin{blockcode}[commandchars=\\\{\}]
\color{gray}/cam/src/chemistry/mozart/mo_usrrxt.F90 (1062,1070)
!-----------------------------------------------------------------
!       ... DMS + OH  --> .5 * SO2
!-----------------------------------------------------------------
       if( usr_DMS_OH_ndx > 0 ) then
          call comp_exp( exp_fac, 7460._r8*tinv, ncol )
          ko(:) = 1._r8 + 5.5e-31_r8 * exp_fac * m(:,k) * 0.21_r8
          call comp_exp( exp_fac, 7810._r8*tinv, ncol )
          rxt(:,k,usr_DMS_OH_ndx) = 1.7e-42_r8 * exp_fac * m(:,k) * 0.21_r8 / ko(:)
       end if
\end{blockcode}

This is equivalent to:

\begin{equation}
k = \frac{1.7 \times 10^{-42} e^{\frac{7810}{T}} [\mbox{M}] 0.21}{1 + 5.5 \times 10^{-31} e^{\frac{7460}{T}} [\mbox{M}] 0.21 }
\end{equation}

This can be rearranged as a Troe reaction,

\begin{equation}
\begin{split}
k & = \frac{k_0[\mbox{M}]}{1+k_0[\mbox{M}]/k_{\inf}}F_C^{(1+1/N[log_{10}(k_0[\mbox{M}]/k_{\inf})]^2)^{-1}} \\
k_0 & = A_0 e^{\left( \frac{C_0}{T} \right)} \left( \frac{T}{300} \right)^{B_0} \\
k_{inf} & = A_{inf} e^{\left( \frac{C_{inf}}{T} \right)} \left( \frac{T}{300} \right)^{B_{inf}}
\end{split}
\end{equation}

\noindent where $F_C = 1$, $A_0 = 0.21 \times 1.7 \times 10^{-42}$, $B_0 = 0$, $C_0 = 7810$, $A_{inf} = \frac{1.7 \times 10^{-42}}{5.5 \times 10^{-31}}$, $B_{inf} = 0$, and $C_{inf} = 7810-7460$.

%%%%%%%%%%%%%%%%%%%%%%%%%%%%%%%%%%%%%%%%%%%%%%%

\section{usr\_GLYOXAL\_aer}

Aerosol surface reaction

%%%%%%%%%%%%%%%%%%%%%%%%%%%%%%%%%%%%%%%%%%%%%%%

\section{usr\_PBZNIT\_M}

\begin{blockcode}[commandchars=\\\{\}]
\color{gray}/cam/src/chemistry/mozart/mo_usrrxt.F90 (971,971)
       call comp_exp( exp_fac, -14000._r8*tinv, ncol )

\color{gray}...

\color{gray}/cam/src/chemistry/mozart/mo_usrrxt.F90 (1006,1015)
!-----------------------------------------------------------------
!       ... pbznit + m --> acbzo2 + no2 + m
!-----------------------------------------------------------------
       if( usr_PBZNIT_M_ndx > 0 ) then
          if( tag_ACBZO2_NO2_ndx > 0 ) then
             rxt(:,k,usr_PBZNIT_M_ndx) = rxt(:,k,tag_ACBZO2_NO2_ndx) * 1.111e28_r8 * exp_fac(:)
          else
             rxt(:,k,usr_PBZNIT_M_ndx) = 0._r8
          end if
       end if
\end{blockcode}

The reaction rates set in \verb>mo_usrrxt::usrrxt()> do not appear to include those for \verb>tag_ACBZO2_NO2_ndx>. As the \verb>rxt(:,:)> array is an \verb>intent(inout)> argument, it could already contain this rate when the function is called. The \verb>mo_usrrxt::usrrxt()> function is called by \verb>mo_gas_phase_chemdr::gas_phase_chemdr()>, which declares and initializes an array called \verb>reaction_rates(:,:,:)> which is passed to \verb>mo_usrrxt::usrrxt()> as \verb>rxt(:,:)>. Prior to calling \verb>mo_usrrxt::usrrxt()>, the functions \verb>ratecon_sfstrat()> and \verb>mo_setrxt::setrxt()> are called, passing \verb>reaction_rates(:,:,:)> as an argument:

\begin{blockcode}[commandchars=\\\{\}]
\color{gray}cam/src/chemistry/mozart/mo_gas_phase_chemdr.F90 (242,252)
!-----------------------------------------------------------------------
!-----------------------------------------------------------------------
  subroutine gas_phase_chemdr(lchnk, ncol, imozart, q, &
                              phis, zm, zi, calday, &
                              tfld, pmid, pdel, pint,  &
                              cldw, troplev, troplevchem, &
                              ncldwtr, ufld, vfld,  &
                              delt, ps, xactive_prates, &
                              fsds, ts, asdir, ocnfrac, icefrac, &
                              precc, precl, snowhland, ghg_chem, latmapback, &
                              drydepflx, wetdepflx, cflx, fire_sflx, fire_ztop, nhx_nitrogen_flx, noy_nitrogen_flx, qtend, pbuf)

\color{gray}...

\color{gray}cam/src/chemistry/mozart/mo_gas_phase_chemdr.F90 (387,387)
    real(r8)     ::  reaction_rates(ncol,pver,max(1,rxntot))      ! reaction rates

\color{gray}...

\color{gray}cam/src/chemistry/mozart/mo_gas_phase_chemdr.F90 (487,488)
    ! initialize to NaN to hopefully catch user defined rxts that go unset
    reaction_rates(:,:,:) = nan

\color{gray}...

\color{gray}cam/src/chemistry/mozart/mo_gas_phase_chemdr.F90 (705,712)
       !-----------------------------------------------------------------------
       !        ... call aerosol reaction rates
       !-----------------------------------------------------------------------
       call ratecon_sfstrat( ncol, invariants(:,:,indexm), pmid, tfld, &
            radius_strat(:,:,1), sad_strat(:,:,1), sad_strat(:,:,2), &
            sad_strat(:,:,3), h2ovmr, vmr, reaction_rates, &
            gprob_n2o5, gprob_cnt_hcl, gprob_cnt_h2o, gprob_bnt_h2o, &
            gprob_hocl_hcl, gprob_hobr_hcl, wtper )

\color{gray}...

\color{gray}cam/src/chemistry/mozart/mo_gas_phase_chemdr.F90 (735,738)
    !-----------------------------------------------------------------------
    !       ...  Set rates for "tabular" and user specified reactions
    !-----------------------------------------------------------------------
    call setrxt( reaction_rates, tfld, invariants(1,1,indexm), ncol )

\color{gray}...

\color{gray}cam/src/chemistry/mozart/mo_gas_phase_chemdr.F90 (774,776)
    call usrrxt( reaction_rates, tfld, ion_temp_fld, ele_temp_fld, invariants, h2ovmr, &
                 pmid, invariants(:,:,indexm), sulfate, mmr, relhum, strato_sad, &
                 troplevchem, dlats, ncol, sad_trop, reff, cwat, mbar, pbuf )
\end{blockcode}

There are a number of different definitions of the \verb>mo_setrxt> module, presumably for different model configurations, which must be chosen in some way during the build process. Each of these modules appears to set different rates, with hard-coded indices and no description of what these rates correspond to. For example, the \verb>mo_setrxt::setrxt()> function sets four rates:

\begin{blockcode}[commandchars=\\\{\}]
\color{gray}components/cam/src/chemistry/pp_trop_mam7/mo_setrxt.F90 (38,41)
      rate(:,:,3) = 2.9e-12_r8 * exp( -160._r8 * itemp(:,:) )
      rate(:,:,5) = 9.6e-12_r8 * exp( -234._r8 * itemp(:,:) )
      rate(:,:,7) = 1.9e-13_r8 * exp( 520._r8 * itemp(:,:) )
      rate(:,:,8) = 1.7e-12_r8 * exp( -710._r8 * itemp(:,:) )
\end{blockcode}

\vspace{30px}
\noindent\textit{\Large Is there a way to know what reaction rates these indices correspond to?}
\vspace{30px}

I will assume that these modules are written by some pre-processor and hope that the rate they set for whatever index corresponds to \verb>tag_ACBZO2_NO2_ndx> is based on the following reaction that appears in the TS1 mechanism:

\begin{blockcode}
        {
          "type": "TROE",
          "k0_A": 9.7e-29,
          "k0_B": -5.6,
          "kinf_A": 9.3e-12,
          "N": 1.5,
          "Fc": 0.6,
          "reactants": {
            "ACBZO2": { },
            "NO2": { },
            "M": { }
          },
          "products": {
            "PBZNIT": { },
            "M": { }
          }
        },
\end{blockcode}

\noindent as this is the only reaction with \verb>ACBZO2> and \verb>NO2> as reactants. If this is the case, the \verb>usr_PBZNIT_M> rate is:

\begin{equation}
\begin{split}
k & = \frac{k_0[\mbox{M}]}{1+k_0[\mbox{M}]/k_{\inf}}F_C^{(1+1/N[log_{10}(k_0[\mbox{M}]/k_{\inf})]^2)^{-1}} 1.111 \times 10^{28} e^{\left(\frac{-14000}{T}\right)}\\
k_0 & = A_0 e^{\left( \frac{C_0}{T} \right)} \left( \frac{T}{300} \right)^{B_0} \\
k_{inf} & = A_{inf} e^{\left( \frac{C_{inf}}{T} \right)} \left( \frac{T}{300} \right)^{B_{inf}}
\end{split}
\end{equation}

\noindent where $F_C = 0.6$, $N = 1.5$, $A_0 = 9.7 \times 10^{-29}$, $B_0 = -5.6$, $C_0 = 0$, $A_{inf} = 9.3 \times 10^{-12}$, $B_{inf} = 0$, and $C_{inf} = 0$. This can be rearranged into a Troe reaction as:

\begin{equation}
\begin{split}
k & = \frac{k_0[\mbox{M}]}{1+k_0[\mbox{M}]/k_{\inf}}F_C^{(1+1/N[log_{10}(k_0[\mbox{M}]/k_{\inf})]^2)^{-1}} \\
k_0 & = A_0 e^{\left( \frac{C_0}{T} \right)} \left( \frac{T}{300} \right)^{B_0} \\
k_{inf} & = A_{inf} e^{\left( \frac{C_{inf}}{T} \right)} \left( \frac{T}{300} \right)^{B_{inf}}
\end{split}
\end{equation}

\noindent where $F_C = 0.6$, $N = 1.5$, $A_0 = 9.7 \times 10^{-29} \times 1.111 \times 10^{28}$, $B_0 = -5.6$, $C_0 = -14000$, $A_{inf} = 9.3 \times 10^{-12} \times 1.111 \times 10^{28}$, $B_{inf} = 0$, and $C_{inf} = -14000$.

%%%%%%%%%%%%%%%%%%%%%%%%%%%%%%%%%%%%%%%%%%%%%%%

\section{usr\_O\_O2}

\begin{blockcode}[commandchars=\\\{\}]
\color{gray}cam/src/chemistry/mozart/mo_usrrxt.F90 (799,807)
       tp(:)             = 300._r8 * tinv(:)

\color{gray}...

\color{gray}cam/src/chemistry/mozart/mo_usrrxt.F90 (799,807)
!-----------------------------------------------------------------
! ... o + o2 + m --> o3 + m (JPL15-10)
!-----------------------------------------------------------------
       if( usr_O_O2_ndx > 0 ) then
          rxt(:,k,usr_O_O2_ndx) = 6.e-34_r8 * tp(:)**2.4_r8
       end if
\end{blockcode}

This is equivalent to:

\begin{equation}
k = 6.0 \times 10^{-34} \left(\frac{300}{T}\right)^{2.4},
\end{equation}

\noindent which can be rearranged into an Arrhenius reaction as:

\begin{equation}
k = Ae^{(\frac{-E_a}{k_bT})}(\frac{T}{D})^B(1.0+E \times P),
\end{equation}

\noindent with $A = 6.0 \times 10^{-34}$, $B = -2.4$, $E_a = 0$, $D = 300$, and $E = 0$.

%%%%%%%%%%%%%%%%%%%%%%%%%%%%%%%%%%%%%%%%%%%%%%%

\section{usr\_HO2\_aer}

Aerosol surface reaction

%%%%%%%%%%%%%%%%%%%%%%%%%%%%%%%%%%%%%%%%%%%%%%%

\section{usr\_N2O5\_M}

\begin{blockcode}[commandchars=\\\{\}]
\color{gray}cam/src/chemistry/mozart/mo_usrrxt.F90 (849,859)
!-----------------------------------------------------------------
! ... n2o5 + m --> no2 + no3 + m (JPL15-10)
!-----------------------------------------------------------------
       if( usr_N2O5_M_ndx > 0 ) then
          if( tag_NO2_NO3_ndx > 0 ) then
             call comp_exp( exp_fac, -10840.0_r8*tinv, ncol )
             rxt(:,k,usr_N2O5_M_ndx) = rxt(:,k,tag_NO2_NO3_ndx) * 1.724138e26_r8 * exp_fac(:)
          else
             rxt(:,k,usr_N2O5_M_ndx) = 0._r8
          end if
       end if
\end{blockcode}

Following the same logic as for \verb>usr_PBZNIT_M>, the \verb>tag_NO2_NO3_ndx> reaction is identified as:

\begin{blockcode}
        {
          "type": "TROE",
          "k0_A": 2.4e-30,
          "k0_B": -3,
          "kinf_A": 1.6e-12,
          "N": -0.1,
          "Fc": 0.6,
          "reactants": {
            "NO3": { },
            "NO2": { },
            "M": { }
          },
          "products": {
            "N2O5": { },
            "M": { }
          }
        },
\end{blockcode}

\noindent and can be rearranged as a Troe reaction:

\begin{equation}
\begin{split}
k & = \frac{k_0[\mbox{M}]}{1+k_0[\mbox{M}]/k_{\inf}}F_C^{(1+1/N[log_{10}(k_0[\mbox{M}]/k_{\inf})]^2)^{-1}} \\
k_0 & = A_0 e^{\left( \frac{C_0}{T} \right)} \left( \frac{T}{300} \right)^{B_0} \\
k_{inf} & = A_{inf} e^{\left( \frac{C_{inf}}{T} \right)} \left( \frac{T}{300} \right)^{B_{inf}}
\end{split}
\end{equation}

\noindent where $F_C = 0.6$, $N = -0.1$, $A_0 = 2.4 \times 10^{-30} \times 1.724138 \times 10^{26}$, $B_0 = -3$, $C_0 = -10840$, $A_{inf} = 1.6 \times 10^{-12} \times 1.724138 \times 10^{26}$, $B_{inf} = 0$, and $C_{inf} = -10840$.

%%%%%%%%%%%%%%%%%%%%%%%%%%%%%%%%%%%%%%%%%%%%%%%

\section{usr\_HO2NO2\_M}

\begin{blockcode}[commandchars=\\\{\}]
\color{gray}cam/src/chemistry/mozart/mo_usrrxt.F90 (898,905)
       if( usr_HO2NO2_M_ndx > 0 ) then
          if( tag_NO2_HO2_ndx > 0 ) then
             call comp_exp( exp_fac, -10900._r8*tinv, ncol )
             rxt(:,k,usr_HO2NO2_M_ndx) = rxt(:,k,tag_NO2_HO2_ndx) * exp_fac(:) / 2.1e-27_r8
          else
             rxt(:,k,usr_HO2NO2_M_ndx) = 0._r8
          end if
       end if
\end{blockcode}

Following the same logic as for \verb>usr_PBZNIT_M>, the \verb>tag_NO2_HO2_ndx> reaction is identified as:

\begin{blockcode}
        {
          "type": "TROE",
          "k0_A": 1.9e-31,
          "k0_B": -3.4,
          "kinf_A": 4e-12,
          "N": 0.3,
          "Fc": 0.6,
          "reactants": {
            "NO2": { },
            "HO2": { },
            "M": { }
          },
          "products": {
            "HO2NO2": { },
            "M": { }
          }
        },
\end{blockcode}

\noindent and can be rearranged as a Troe reaction:

\begin{equation}
\begin{split}
k & = \frac{k_0[\mbox{M}]}{1+k_0[\mbox{M}]/k_{\inf}}F_C^{(1+1/N[log_{10}(k_0[\mbox{M}]/k_{\inf})]^2)^{-1}} \\
k_0 & = A_0 e^{\left( \frac{C_0}{T} \right)} \left( \frac{T}{300} \right)^{B_0} \\
k_{inf} & = A_{inf} e^{\left( \frac{C_{inf}}{T} \right)} \left( \frac{T}{300} \right)^{B_{inf}}
\end{split}
\end{equation}

\noindent where $F_C = 0.6$, $N = 0.3$, $A_0 = 1.9 \times 10^{-31} / \left( 2.1 \times 10^{-27} \right)$, $B_0 = -3.4$, $C_0 = -10900$, $A_{inf} = 4 \times 10^{-12} / \left( 2.1 \times 10^{-27} \right)$, $B_{inf} = 0$, and $C_{inf} = -10900$.

%%%%%%%%%%%%%%%%%%%%%%%%%%%%%%%%%%%%%%%%%%%%%%%

\section{usr\_NO3\_aer}

Aerosol surface reaction

%%%%%%%%%%%%%%%%%%%%%%%%%%%%%%%%%%%%%%%%%%%%%%%

\section{usr\_HO2\_HO2}

\begin{blockcode}[commandchars=\\\{\}]
\color{gray}cam/src/chemistry/mozart/mo_usrrxt.F90 (2212,2219)
!-----------------------------------------------------------------
! ... ho2 + ho2 --> h2o2
! Note: this rate involves the water vapor number density
!-----------------------------------------------------------------
         ko(:)   = 3.0e-13_r8  * exp( 460._r8*tinv(:) )
         kinf(:) = 2.1e-33_r8 * m(:,k) * exp( 920._r8*tinv(:) )
         fc(:)   = 1._r8 + 1.4e-21_r8 * m(:,k) * h2ovmr(:,k) * exp( 2200._r8*tinv(:) )
         rxt(:,k,usr_HO2_HO2_ndx) = (ko(:) + kinf(:)) * fc(:)
\end{blockcode}

The term \verb>m(:,k) * h2ovmr(:,k)> should be the water vapor number density (assuming \verb>h2ovmr(:,k)> is in units of $\mbox{mol}/\mbox{mol}$). This reaction can then be rearranged as four reactions:
\vspace{20px}


\ch{HO2 + HO2 -> H2O2}

\ch{HO2 + HO2 ->[ M ] H2O2}

\ch{HO2 + HO2 ->[ H2O ] H2O2}

\ch{HO2 + HO2 ->[ M, H2O ] H2O2}

\vspace{20px}

\noindent with Arrhenius rate constants:

\begin{equation}
k = Ae^{(\frac{C}{T})}(\frac{T}{D})^B(1.0+E \times P),
\end{equation}

\noindent with rate constant parameters, respectively:

\begin{itemize}
\item $A = 3.0 \times 10^{-13}$, $B = 0$, $C = 460$, and $E = 0$.
\item $A = 2.1 \times 10^{-33}$, $B = 0$, $C = 920$, and $E = 0$.
\item $A = 3.0 \times 10^{-13} \times 1.4 \times 10^{-21}$, $B = 0$, $C = 2660$, and $E = 0$.
\item $A = 2.1 \times 10^{-33} \times 1.4 \times 10^{-21}$, $B = 0$, $C = 3120$, and $E = 0$.
\end{itemize}

%%%%%%%%%%%%%%%%%%%%%%%%%%%%%%%%%%%%%%%%%%%%%%%

\section{usr\_MPAN\_M}

\begin{blockcode}[commandchars=\\\{\}]
\color{gray}cam/src/chemistry/mozart/mo_usrrxt.F90 (971,971)
       call comp_exp( exp_fac, -14000._r8*tinv, ncol )
\end{blockcode}
\begin{blockcode}[commandchars=\\\{\}]
\color{gray}cam/src/chemistry/mozart/mo_usrrxt.F90 (987,996)
!-----------------------------------------------------------------
! ... mpan + m --> mco3 + no2 + m (JPL15-10)
!-----------------------------------------------------------------
       if( usr_MPAN_M_ndx > 0 ) then
          if( tag_MCO3_NO2_ndx > 0 ) then
             rxt(:,k,usr_MPAN_M_ndx) = rxt(:,k,tag_MCO3_NO2_ndx) * 1.111e28_r8 * exp_fac(:)
          else
             rxt(:,k,usr_MPAN_M_ndx) = 0._r8
          end if
       end if
\end{blockcode}

This rate constant calculation appears to be based on the reaction:
\vspace{20px}

\ch{MCO3 + NO2 ->[ M ] MPAN}

\vspace{20px}
\noindent and this reaction in the mechanism has a custom rate constant:

\begin{verbatim}
    {
      "type": "UNSUPPORTED",
      "label": "usr_MCO3_NO2",
      "reactants": {
        "MCO3": { },
        "NO2": { },
        "M": { }
      },
      "products": {
        "MPAN": { },
        "M": { }
      }
    },
\end{verbatim}

\noindent however, the \verb>usr_MPAN_M> reaction uses \verb>tag_MCO3_NO2_ndx> instead of \verb>usr_MCO3_NO2_ndx>, which I believe indicates that this is a standard (Arrhenius or Troe) reaction rather than a custom reaction. Looking through the code, I see that this reaction does appear to sometimes be treated as a Troe reaction:

\begin{blockcode}[commandchars=\\\{\}]
\color{gray}components/cam/src/chemistry/pp_trop_strat_mam4_ts2/chem_mech.doc (1147,1149)
  tag_MCO3_NO2     (240)   MCO3 + NO2 + M ->  MPAN + M                                          troe : ko=9.70E-29*(300/t)**5.60  (407)
                                                                                                       ki=9.30E-12*(300/t)**1.50
                                                                                                        f=0.60
\end{blockcode}

After discussion with Louisa and Becky, the Troe reaction for \ch{MCO3 + NO2 ->[M] MPAN} is the correct one to use, with $F_c = 0.6$, $N = 1.5$, $A_0 = 9.7 \times 10^{-29}$, $B_0 = -5.6$, $C_0 = 0$, $A_{inf} = 9.3 \times 10^{-12}$, $B_{inf} = 0$, and $C_{inf} = 0$.

This means the \verb>usr_MPAN_M> reaction can be rearranged as a Troe reaction with $F_c = 0.6$, $N = 1.5$, $A_0 = 9.7 \times 10^{-29} \times 1.111 \times 10^{28}$, $B_0 = -5.6$, $C_0 = -14000$, $A_{inf} = 9.3 \times 10^{-12} \times 1.111 \times 10^{28}$, $B_{inf} = 0$, and $C_{inf} = -14000$.

%%%%%%%%%%%%%%%%%%%%%%%%%%%%%%%%%%%%%%%%%%%%%%%

\section{usr\_SO2\_OH}

\begin{blockcode}[commandchars=\\\{\}]
\color{gray}cam/src/chemistry/mozart/mo_usrrxt.F90 (1072,1079)
!-----------------------------------------------------------------
!       ... SO2 + OH  --> SO4  (REFERENCE?? - not Liao)
!-----------------------------------------------------------------
       if( usr_SO2_OH_ndx > 0 ) then
          fc(:) = 3.0e-31_r8 *(300._r8*tinv(:))**3.3_r8
          ko(:) = fc(:)*m(:,k)/(1._r8 + fc(:)*m(:,k)/1.5e-12_r8)
          rxt(:,k,usr_SO2_OH_ndx) = ko(:)*.6_r8**(1._r8 + (log10(fc(:)*m(:,k)/1.5e-12_r8))**2._r8)**(-1._r8)
       end if
\end{blockcode}

This is a Troe reaction:

\begin{equation}
\begin{split}
k & = \frac{k_0[\mbox{M}]}{1+k_0[\mbox{M}]/k_{\inf}}F_C^{(1+1/N[log_{10}(k_0[\mbox{M}]/k_{\inf})]^2)^{-1}} \\
k_0 & = A_0 e^{\left( \frac{C_0}{T} \right)} \left( \frac{T}{300} \right)^{B_0} \\
k_{inf} & = A_{inf} e^{\left( \frac{C_{inf}}{T} \right)} \left( \frac{T}{300} \right)^{B_{inf}}
\end{split}
\end{equation}

\noindent where $F_C = 0.6$, $N = 1.0$, $A_0 = 3 \times 10^{-31}$, $B_0 = -3.3$, $C_0 = 0$, $A_{inf} = 1.5 \times 10^{-12}$, $B_{inf} = 0$, and $C_{inf} = 0$.

The parameters for this reaction appear to come from DeMore et al. 1997 (\url{https://jpldataeval.jpl.nasa.gov/pdf/Atmos97_Anotated.pdf} page 129).


%%%%%%%%%%%%%%%%%%%%%%%%%%%%%%%%%%%%%%%%%%%%%%%

\section{usr\_CO\_OH\_a}

\begin{blockcode}[commandchars=\\\{\}]
\color{gray}cime/src/share/util/shr_const_mod.F90 (24,24)
   real(R8),parameter :: SHR_CONST_BOLTZ   = 1.38065e-23_R8  ! Boltzmann's constant ~ J/K/molecule
\end{blockcode}
\begin{blockcode}[commandchars=\\\{\}]
\color{gray}cam/src/utils/physconst.F90 (60,60)
real(r8), public, parameter :: boltz       = shr_const_boltz      ! Boltzman's constant (J/K/molecule)
\end{blockcode}
\begin{blockcode}[commandchars=\\\{\}]
\color{gray}cam/src/chemistry/utils/mo_constants.F90 (25,25)
  real(r8), parameter ::  boltz_cgs   = boltz*1.e7_r8       ! erg/K
\end{blockcode}
\begin{blockcode}[commandchars=\\\{\}]
\color{gray}cam/src/chemistry/mozart/mo_usrrxt.F90 (914,920)
!-----------------------------------------------------------------
!           co + oh --> co2 + ho2     (combined branches - do not use with CO_OH_b)
!-----------------------------------------------------------------
       if( usr_CO_OH_a_ndx > 0 ) then
          rxt(:,k,usr_CO_OH_a_ndx) = 1.5e-13_r8 * &
               (1._r8 + 6.e-7_r8*boltz_cgs*m(:,k)*temp(:ncol,k))
       end if
\end{blockcode}

This can be split into two reactions:
\vspace{20px}

\ch{CO + OH -> CO2 + HO2}

\ch{CO + OH ->[ M ] CO2 + HO2}

\vspace{20px}

\noindent with Arrhenius rate constants:

\begin{equation}
k = Ae^{(\frac{C}{T})}(\frac{T}{D})^B(1.0+E \times P),
\end{equation}

\noindent with rate constant parameters, respectively:

\begin{itemize}
\item $A = 1.5 \times 10^{-13}$, $B = 0$, $C = 0$, and $E = 0$.
\item $A = 1.5 \times 10^{-13} \times 6 \times 10^{-7} \times k_B$, $B = 1$, $C = 0$, $D = 1$, and $E = 0$.
\end{itemize}

\noindent where $k_B$ is the Boltzmann constant [erg $\mbox{K}^{-1}$]. \textbf{Watch out for the non-SI unit erg.}


%%%%%%%%%%%%%%%%%%%%%%%%%%%%%%%%%%%%%%%%%%%%%%%

\section{usr\_O\_O}

\begin{blockcode}[commandchars=\\\{\}]
\color{gray}cam/src/chemistry/mozart/mo_usrrxt.F90 (810,815)
!-----------------------------------------------------------------
! ... o + o + m -> o2 + m
!-----------------------------------------------------------------
       if ( usr_O_O_ndx > 0 ) then
          rxt(:,k,usr_O_O_ndx) = 2.76e-34_r8 * exp( 720.0_r8*tinv(:) )
       end if
\end{blockcode}

This is an Arrhenius reaction:

\begin{equation}
k = Ae^{(\frac{C}{T})}(\frac{T}{D})^B(1.0+E \times P),
\end{equation}

\noindent with $A = 2.76 \times 10^{-34}$, $B = 0$, $C = 720$, and $E = 0$.
 
 
%%%%%%%%%%%%%%%%%%%%%%%%%%%%%%%%%%%%%%%%%%%%%%%
          
\section{usr\_N2O5\_aer}

Aerosol surface reaction

%%%%%%%%%%%%%%%%%%%%%%%%%%%%%%%%%%%%%%%%%%%%%%%

\section{usr\_NO2\_aer}

Aerosol surface reaction

%%%%%%%%%%%%%%%%%%%%%%%%%%%%%%%%%%%%%%%%%%%%%%%

\section{usr\_PAN\_M}

\begin{blockcode}[commandchars=\\\{\}]
\color{gray}cam/src/chemistry/mozart/mo_usrrxt.F90 (968,978)
!-----------------------------------------------------------------
! ... pan + m --> ch3co3 + no2 + m (JPL15-10)
!-----------------------------------------------------------------
       call comp_exp( exp_fac, -14000._r8*tinv, ncol )
       if( usr_PAN_M_ndx > 0 ) then
          if( tag_CH3CO3_NO2_ndx > 0 ) then
             rxt(:,k,usr_PAN_M_ndx) = rxt(:,k,tag_CH3CO3_NO2_ndx) * 1.111e28_r8 * exp_fac(:)
          else
             rxt(:,k,usr_PAN_M_ndx) = 0._r8
          end if
       end if
\end{blockcode}

Following the same logic as for \verb>usr_PBZNIT_M>, the \verb>tag_CH3CO3_NO2_ndx> reaction is identified as:

\begin{blockcode}
         {
          "type": "TROE",
          "k0_A": 9.7e-29,
          "k0_B": -5.6,
          "kinf_A": 9.3e-12,
          "N": 1.5,
          "Fc": 0.6,
          "reactants": {
            "CH3CO3": { },
            "NO2": { },
            "M": { }
          },
          "products": {
            "PAN": { },
            "M": { }
          }
        },
\end{blockcode}

\noindent and can be rearranged as a Troe reaction:

\begin{equation}
\begin{split}
k & = \frac{k_0[\mbox{M}]}{1+k_0[\mbox{M}]/k_{\inf}}F_C^{(1+1/N[log_{10}(k_0[\mbox{M}]/k_{\inf})]^2)^{-1}} \\
k_0 & = A_0 e^{\left( \frac{C_0}{T} \right)} \left( \frac{T}{300} \right)^{B_0} \\
k_{inf} & = A_{inf} e^{\left( \frac{C_{inf}}{T} \right)} \left( \frac{T}{300} \right)^{B_{inf}}
\end{split}
\end{equation}

\noindent where $F_C = 0.6$, $N = 1.5$, $A_0 = 9.7 \times 10^{-29} \times 1.111 \times 10^{28}$, $B_0 = -5.6$, $C_0 = -14000$, $A_{inf} = 9.3 \times 10^{-12} \times 1.111 \times 10^{28}$, $B_{inf} = 0$, and $C_{inf} = -14000$.

%%%%%%%%%%%%%%%%%%%%%%%%%%%%%%%%%%%%%%%%%%%%%%%

\section{usr\_HNO3\_OH}

\begin{blockcode}[commandchars=\\\{\}]
\color{gray}cam/src/chemistry/mozart/mo_usrrxt.F90 (877,889)
!-----------------------------------------------------------------
! set rates for:
!   ... hno3 + oh --> no3 + h2o
!           ho2no2 + m --> ho2 + no2 + m
!-----------------------------------------------------------------
       if( usr_HNO3_OH_ndx > 0 ) then
          call comp_exp( exp_fac, 1335._r8*tinv, ncol )
          ko(:) = m(:,k) * 6.5e-34_r8 * exp_fac(:)
          call comp_exp( exp_fac, 2199._r8*tinv, ncol )
          ko(:) = ko(:) / (1._r8 + ko(:)/(2.7e-17_r8*exp_fac(:)))
          call comp_exp( exp_fac, 460._r8*tinv, ncol )
          rxt(:,k,usr_HNO3_OH_ndx) = ko(:) + 2.4e-14_r8*exp_fac(:)
       end if
\end{blockcode}

This can be split into two reactions with the same reactants and products:
\vspace{20px}

\ch{HNO3 + OH -> NO3 + H2O}

\vspace{20px}
\noindent the first with an Arrhenius rate constant:

\begin{equation}
k = Ae^{(\frac{C}{T})}(\frac{T}{D})^B(1.0+E \times P),
\end{equation}

\noindent with $A = 2.4 \times 10^{-14}$, $B = 0$, $C = 460$, and $E = 0$, and the second
with a Troe rate constant:

\begin{equation}
\begin{split}
k & = \frac{k_0[\mbox{M}]}{1+k_0[\mbox{M}]/k_{\inf}}F_C^{(1+1/N[log_{10}(k_0[\mbox{M}]/k_{\inf})]^2)^{-1}} \\
k_0 & = A_0 e^{\left( \frac{C_0}{T} \right)} \left( \frac{T}{300} \right)^{B_0} \\
k_{inf} & = A_{inf} e^{\left( \frac{C_{inf}}{T} \right)} \left( \frac{T}{300} \right)^{B_{inf}}
\end{split}
\end{equation}

\noindent where $F_C = 1$, $A_0 = 6.5 \times 10^{-34}$, $B_0 = 0$, $C_0 = 1335$, $A_{inf} = 2.7 \times 10^{-17}$, $B_{inf} = 0$, and $C_{inf} = 2199$.


%%%%%%%%%%%%%%%%%%%%%%%%%%%%%%%%%%%%%%%%%%%%%%%

\section{usr\_MCO3\_NO2}

\begin{blockcode}[commandchars=\\\{\}]
\color{gray}cam/src/chemistry/mozart/mo_usrrxt.F90 (804,804)
       tp(:)             = 300._r8 * tinv(:)
\color{gray}cam/src/chemistry/mozart/mo_usrrxt.F90 (958,996)
!-----------------------------------------------------------------
!     ... mco3 + no2 -> mpan
!-----------------------------------------------------------------
       if( usr_MCO3_NO2_ndx > 0 ) then
          rxt(:,k,usr_MCO3_NO2_ndx) = 1.1e-11_r8 * tp(:) / m(:,k)
       end if
       if( usr_MCO3_XNO2_ndx > 0 ) then
          rxt(:,k,usr_MCO3_XNO2_ndx) = 1.1e-11_r8 * tp(:) / m(:,k)
       end if
\end{blockcode}

This can be rearranged as an Arrhenius reaction, \textbf{by removing M from the reactants and products}:

\begin{equation}
k = Ae^{(\frac{C}{T})}(\frac{T}{D})^B(1.0+E \times P),
\end{equation}

\noindent with $A = 1.1 \times 10^{-11}$, $B = -1$, $C = 0$, $D = 300$, and $E = 0$.

%%%%%%%%%%%%%%%%%%%%%%%%%%%%%%%%%%%%%%%%%%%%%%%

\section{usr\_CH3COCH3\_OH}

\begin{blockcode}[commandchars=\\\{\}]
\color{gray}cam/src/chemistry/mozart/mo_usrrxt.F90 (1054,1060)
!-----------------------------------------------------------------
!       ... ch3coch3 + oh -> ro2 + h2o
!-----------------------------------------------------------------
       if( usr_CH3COCH3_OH_ndx > 0 ) then
          call comp_exp( exp_fac, -2000._r8*tinv, ncol )
          rxt(:,k,usr_CH3COCH3_OH_ndx) = 3.82e-11_r8 * exp_fac(:) + 1.33e-13_r8
       end if
\end{blockcode}

This can be split into two reactions with the same products and reactants:
\vspace{20px}

\ch{CH3COCH3 + OH -> RO2 + H2O}

\vspace{20px}

\noindent both with Arrhenius rate constants:

\begin{equation}
k = Ae^{(\frac{C}{T})}(\frac{T}{D})^B(1.0+E \times P),
\end{equation}

\noindent with rate constant parameters:

\begin{itemize}
  \item $A = 3.82 \times 10^{-11}$, $B = 0$, $C = -2000$, and $E = 0$.
  \item $A = 1.33 \times 10^{-13}$, $B = 0$, $C = 0$, and $E = 0$.
\end{itemize}

%%%%%%%%%%%%%%%%%%%%%%%%%%%%%%%%%%%%%%%%%%%%%%%

\section{usr\_CL2O2\_M}

\begin{blockcode}[commandchars=\\\{\}]
\color{gray}cam/src/chemistry/mozart/mo_usrrxt.F90 (817,827)
!-----------------------------------------------------------------
!   ... cl2o2 + m -> 2*clo + m  (JPL15-10)
!-----------------------------------------------------------------
       if ( usr_CL2O2_M_ndx > 0 ) then
          if ( tag_CLO_CLO_M_ndx > 0 ) then
             ko(:)            = 2.16e-27_r8 * exp( 8537.0_r8* tinv(:) )
             rxt(:,k,usr_CL2O2_M_ndx) = rxt(:,k,tag_CLO_CLO_M_ndx)/ko(:)
          else
             rxt(:,k,usr_CL2O2_M_ndx) = 0._r8
          end if
       end if
\end{blockcode}

Following the same logic as for \verb>usr_PBZNIT_M>, the \verb>tag_CLO_CLO_M_ndx> reaction is identified as:

\begin{blockcode}
        {
          "type": "ARRHENIUS",
          "A": 3e-11,
          "Ea": -3.38259005E-20,
          "reactants": {
            "CLO": { "qty": 2 }
          },
          "products": {
            "CL": { "yield": 2 },
            "O2": { }
          }
        },
\end{blockcode}

\noindent and can be rearranged as an Arrhenius reaction:

\begin{equation}
\begin{split}
k = Ae^{(\frac{C}{T})}(\frac{T}{D})^B(1.0+E \times P),
\end{split}
\end{equation}

\noindent where $A = 3 \times 10^{-11} / \left(2.16 \times 10^{-27}\right)$, $B = 0$, $C = 2450 - 8537$, and $E = 0$.

%%%%%%%%%%%%%%%%%%%%%%%%%%%%%%%%%%%%%%%%%%%%%%%

\section{usr\_SO3\_H2O}

\begin{blockcode}[commandchars=\\\{\}]
\color{gray}cam/src/chemistry/mozart/mo_usrrxt.F90 (829,847)
!-----------------------------------------------------------------
!       ... so3 + 2*h2o --> h2so4 + h2o
!       Note: this reaction proceeds by the 2 intermediate steps below
!           so3 + h2o --> adduct
!           adduct + h2o --> h2so4 + h2o
!               (Lovejoy et al., JCP, pp. 19911-19916, 1996)
! The first order rate constant used here is recommended by JPL 2011.
! This rate involves the water vapor number density.
!-----------------------------------------------------------------

       if ( usr_SO3_H2O_ndx > 0 ) then
          call comp_exp( exp_fac, 6540.0_r8*tinv(:), ncol )
          if( h2o_ndx > 0 ) then
             fc(:) = 8.5e-21_r8 * m(:,k) * h2ovmr(:,k) * exp_fac(:)
          else
             fc(:) = 8.5e-21_r8 * invariants(:,k,inv_h2o_ndx) * exp_fac(:)
          end if
          rxt(:,k,usr_SO3_H2O_ndx) = 1.0e-20_r8 * fc(:)
       end if
\end{blockcode}

As for \verb>usr_HO2_HO2>, the term \verb>m(:,k) * h2ovmr(:,k)> is taken to be the water vapor number density. Interestingly, the use of the \verb>invariants> array here when $\mbox{h2o\_ndx} <= 0$ is not used in the calculation of \verb>usr_HO2_HO2>.
\vspace{20px}

\textit{\Large Should the invariants array be used for the water number density whenever $\mbox{h2o\_ndx} <= 0$ ?}

\vspace{20px}
This can be rearranged as:
\vspace{20px}

\ch{SO3 + H2O ->[H2O] H2SO4}

\vspace{20px}
\noindent with an Arrhenius rate constant:

\begin{equation}
\begin{split}
k = Ae^{(\frac{C}{T})}(\frac{T}{D})^B(1.0+E \times P),
\end{split}
\end{equation}

\noindent where $A = 8.5 \times 10^{-41}$, $B = 0$, $C = 6540$, and $E = 0$.


%%%%%%%%%%%%%%%%%%%%%%%%%%%%%%%%%%%%%%%%%%%%%%%

\section{usr\_CO\_OH\_b}

\begin{blockcode}[commandchars=\\\{\}]
\color{gray}cam/src/chemistry/mozart/mo_usrrxt.F90 (227,227)
  real(r8), parameter :: t0     = 300._r8                ! K
\end{blockcode}
\begin{blockcode}[commandchars=\\\{\}]
\color{gray}cam/src/chemistry/mozart/mo_usrrxt.F90 (829,847)
!-----------------------------------------------------------------
!   ... co + oh --> co2 + h (second branch JPL15-10, with CO+OH+M)
!-----------------------------------------------------------------
       if( usr_CO_OH_b_ndx > 0 ) then
         kinf(:)  = 2.1e+09_r8 * (temp(:ncol,k)/ t0)**(6.1_r8)
         ko  (:)  = 1.5e-13_r8

         term1(:) = ko(:) / ( (kinf(:) / m(:,k)) )
         term2(:) = ko(:) / (1._r8 + term1(:))

         term1(:) = log10( term1(:) )
         term1(:) = 1.0_r8 / (1.0_r8 + term1(:)*term1(:))

         rxt(:ncol,k,usr_CO_OH_b_ndx) = term2(:) * (0.6_r8)**term1(:)
       end if
\end{blockcode}

From JPL15-10, the second branch is:
\vspace{20px}

\ch{OH + CO ->[M] H + CO2}

\vspace{20px}
\noindent which ``proceed[s] via bound intermediates. For example, the reaction between HO and CO to yield \ch{H + CO2} takes place on a potential energy surface that contains the radical HOCO. The yield of H and \ch{CO2} is diminished as the pressure rises. The loss of reactants is thus the sum of two processes, an association to yield HOCO and the chemical activation process yielding H and CO2. The total rate constant for loss of reactants is fit by the equation above for the association added to the chemical activation rate constant.''

The resulting rate constant equation is a Troe rate constant without $[\mbox{M}]$ in the numerator of the first term:

\begin{equation}
\begin{split}
k & = \frac{k_0}{1+k_0[\mbox{M}]/k_{\inf}}F_C^{(1+1/N[log_{10}(k_0[\mbox{M}]/k_{\inf})]^2)^{-1}} \\
k_0 & = A_0 e^{\left( \frac{C_0}{T} \right)} \left( \frac{T}{300} \right)^{B_0} \\
k_{inf} & = A_{inf} e^{\left( \frac{C_{inf}}{T} \right)} \left( \frac{T}{300} \right)^{B_{inf}}
\end{split}
\end{equation}

This reaction type has been added to Music Box as ``\verb>TERNARY_CHEMICAL_ACTIVATION>.'' For this reaction, the rate constant parameters are: $A_0 = 1.5 \times 10^{13}$, $B_0 = 0$, $C_0 = 0$, $A_{inf} = 2.1 \times 10^9$, $B_{inf} = 6.1$, $C_{inf} = 0$, $F_c = 0.6$, and $N = 1$.

%%%%%%%%%%%%%%%%%%%%%%%%%%%%%%%%%%%%%%%%%%%%%%%
%%%%%%%%%%%%%%%%%%%%%%%%%%%%%%%%%%%%%%%%%%%%%%%
%%%%%%%%%%%%%%%%%%%%%%%%%%%%%%%%%%%%%%%%%%%%%%%

\end{document}