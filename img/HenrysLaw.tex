\documentclass[11pt, oneside]{article}   	% use "amsart" instead of "article" for AMSLaTeX format
\usepackage{geometry}                		% See geometry.pdf to learn the layout options. There are lots.
\geometry{letterpaper}                   		% ... or a4paper or a5paper or ... 
%\geometry{landscape}                		% Activate for for rotated page geometry
%\usepackage[parfill]{parskip}    		% Activate to begin paragraphs with an empty line rather than an indent
\usepackage{graphicx}				% Use pdf, png, jpg, or eps§ with pdflatex; use eps in DVI mode
								% TeX will automatically convert eps --> pdf in pdflatex		
\usepackage{amssymb}

\title{Effective Henry's Law Constant}
%\author{The Author}
\date{}							% Activate to display a given date or no date

\begin{document}
\maketitle
\section*{Henry's Law}
The amount of a gas, $X,$ in equilibrium with aqueous-phase concentration of the gas, $X_{\mbox{aq}},$ is proportional to the gas-phase partial pressure, $pX$.
\begin{eqnarray*}
X_{\mbox{aq}} &=& K_H \;\; pX \\
\left[K_H\right] &=& \mbox{moles} X / \mbox{liter of air} / \mbox{atm} \\
\left[X_{\mbox{aq}}\right] &=& \mbox{moles} X / \mbox{liter of liquid } \mbox{H}_2\mbox{O} \\
\left[p_X\right] &=& \mbox{atm}
\end{eqnarray*}
where $K_H$ is the Henry's Law constant.
The temperature dependence of the Henry's Law constant is often parameterized.
\begin{eqnarray*}
K_H = \mbox{kh}_{298} \; \exp\left(\mbox{dh}_r \;\left(\frac{1}{T}-\frac{1}{298}\right)\right) 
\end{eqnarray*}

\section*{Acid}
For an acid, where the compound and its anions are in equilibrium,
\begin{eqnarray*}
H_{eff} &=& K_H \left(1 + \frac{K_1}{[H+]}\left(1 + \frac{K_2}{ [H+]}\right)\right) \\
K_1&=&\mbox{k1}_{298} \; \exp(\mbox{dh1}_r \;(\frac{1}{T}-\frac{1}{298})) \\ 
K_2&=&\mbox{k2}_{298} \; \exp(\mbox{dh2}_r \;(\frac{1}{T}-\frac{1}{298})) \\ 
{[H+]} &=& 10^{-pH}
\end{eqnarray*}


\section*{Base}
For a base, where the compound and its cations are in equilibrium
\begin{eqnarray*}
H_{eff} &=& K_H \left(1 + \frac{K_1}{K_w}{[H+]}\right) \\
K_1&=&\mbox{k1}_{298} \; \exp(\mbox{dh1}_r \;(\frac{1}{T}-\frac{1}{298})) \\ 
K_w&=&\mbox{kw}_{298} \; \exp(\mbox{dh2}_w \;(\frac{1}{T}-\frac{1}{298})) \\ 
K_w &=& \left[ H^+ \right] \left[ OH^-\right]
\end{eqnarray*}

\section*{Example Derivation of $H_{eff}$ for an acid}
Assume the compound and its anions are in equilibrium. 
Define a family for that species, in this case, $CO_2$
For example, 
\begin{eqnarray*}
C(IV) = H_2CO_3 + HCO_3^- + CO_3^=
\end{eqnarray*}
Find the effective Henry's Law for that family, 
\begin{eqnarray*}
H_{eff} = \frac{\left[C(IV)\right]}{p_{CO_2}}
\end{eqnarray*} 
where $p_{CO_2}$ is the partial pressure of $CO_2$. Based on the equilibria, 
\begin{eqnarray*}
\left[H_2CO_3\right] &=& K_H \; p_{CO_2}   \\[3pt]
\left[HCO_3^-\right] &=& K_1 \; \frac{\left[H_2CO_3\right]}{\left[H+\right]} \\[3pt]
\left[CO_3^=\right] &=& K_2 \; \frac{\left[HCO_3-\right]}{\left[H+\right]}
\end{eqnarray*}
substitute those equilibria into the $C(IV)$ equation giving
\begin{eqnarray*}
H_{eff} &=& \left( K_H  p_{CO_2} + K_1  \frac{[H_2CO_3]}{[H+]} + K_2  \frac{[HCO_3^-]}{[H+]} \right)/p_{CO_2}   
\end{eqnarray*}
and further substitution gives
\begin{eqnarray*}
H_{eff} &=& K_H + K_H  \frac{K_1}{\left[H+\right]} + K_H\frac{K_1 K_2}{[H+]^2} 
\end{eqnarray*}
resulting in
\begin{eqnarray*}
H_{eff} &=& K_H  \left(1+  \frac{K_1}{[H+]}  \left(1 +   \frac{K_2}{[H+]}\right)\right) 
\end{eqnarray*}

\section*{Example derivation of $H_{eff}$ for a base}
For a gas that hydrolyzes and dissociates into a cation such as $NH_3$, 
\begin{eqnarray*}
NH_3\mbox{aq} = K_H  \; pNH_3
\end{eqnarray*}
$NH_3$ hydrolyzes to make $NH_3-H_2O = NH_4OH$ which dissociates:
\begin{eqnarray*}
NH_4OH &\leftrightarrow& NH_4^+  + OH^- \\
K_1 &=&  \left[NH4^+\right]\left[OH^-\right]/\left[NH_4OH\right]
\end{eqnarray*}
with an equilibrium constant $K_1$.  

Water also dissociates
\begin{eqnarray*}
H_2O &\leftrightarrow& H^+  + OH^- \\
K_w &=& \left[H^+\right]\left[OH^-\right]
\end{eqnarray*}
with an equilibrium constant $K_w$. 

The algebraic derivation follows as:
\begin{eqnarray*}
\left[ NH_4OH \right] &=& K_H \;\; pNH3 \\
\left[NH_4^+\right] \left[ OH^- \right] &=& K_1 \left[ NH_4OH \right] \\
\therefore \left[ NH_4^+ \right] &=& K_1 \left[ H^+ \right] \left[ NH_4OH \right ]/ K_w 
\end{eqnarray*}
Or using the same derivation as the acid above, for $NH_3$ define the family, $N(-III)$, as 
\begin{eqnarray*}
N(-III) &=& NH_4OH + NH4^+ 
\end{eqnarray*}
It follows that 
\begin{eqnarray*}
N(-III) &=& K_H \;\; pNH_3 + \frac{K_1  }{ \left[OH^-\right]} \left[ NH_4OH \right]\\
N(-III) &=& K_H \;\; pNH_3 + \frac{K_1*\left[H+\right]}{K_w} \; K_H \;\; pNH_3  
\end{eqnarray*}
but by the definition $\left[N(-III)\right] = H_{eff} \; \;pNH_3$ we can identify
\begin{eqnarray*}
H_{eff} = K_H  \left(1 + \frac{K_1 }{K_w}\;\left[H^+\right]\right)
\end{eqnarray*}

\end{document}  