\documentclass[11pt, oneside]{article}   	% use "amsart" instead of "article" for AMSLaTeX format
\usepackage{geometry}                		% See geometry.pdf to learn the layout options. There are lots.
\geometry{letterpaper}                   		% ... or a4paper or a5paper or ... 
%\geometry{landscape}                		% Activate for for rotated page geometry
%\usepackage[parfill]{parskip}    		% Activate to begin paragraphs with an empty line rather than an indent
\usepackage{graphicx}				% Use pdf, png, jpg, or eps§ with pdflatex; use eps in DVI mode
								% TeX will automatically convert eps --> pdf in pdflatex		
\usepackage{amssymb}

\title{Henry's Law}
%\author{The Author}
\date{}							% Activate to display a given date or no date

\begin{document}
\maketitle
\section*{Acid}
% Henry's Law Coefficients are written as Heff = KH + KH K1/[H+] + KH K1 K2/[H+]^2
%                              where KH=kh_298*exp(dh_r*(1/T-1/298))
%                              where K1=k1_298*exp(dh1_r*(1/T-1/298))
%                              where K2=k2_298*exp(dh2_r*(1/T-1/298))
% [H+] is the hydrogen ion concentration that is obtained from the pH of the drop. pH = -log10([H+])
\begin{eqnarray*}
H_{eff} &=& K_H \left(1 + \frac{K_1}{[H+]}\left(1 + \frac{K_2 }{ [H+]}\right)\right)
\end{eqnarray*}
\section*{Base}
\begin{eqnarray*}
H_{eff} &=& K_H \left(1 + \frac{K_1}{K_2}{[H+]}\right)
\end{eqnarray*}
\section*{Where}
\begin{eqnarray*}
K_H&=&\mbox{kh}_{298} \; \exp(\mbox{dh}_r \;(\frac{1}{T}-\frac{1}{298})) \\ [3pt]
K_1&=&\mbox{k1}_{298} \; \exp(\mbox{dh1}_r \;(\frac{1}{T}-\frac{1}{298})) \\ [3pt]
K_2&=&\mbox{k2}_{298} \; \exp(\mbox{dh2}_r \;(\frac{1}{T}-\frac{1}{298})) \\ [3pt]
{[H+]} &=& 10^{-pH}
\end{eqnarray*}

\section*{Example Derivation of $H_{eff}$ for $CO_2$}
Assume the compound and its anions are in equilibrium. 
Define a family for that species. 
For example, 
\begin{eqnarray*}
C(IV) = H_2CO_3 + HCO_3^- + CO_3^=
\end{eqnarray*}
Find the effective Henry's Law for that family, 
\begin{eqnarray*}
H_{eff} = \frac{\left[C(IV)\right]}{p_{CO_2}}
\end{eqnarray*} 
where $p_{CO_2}$ is the partial pressure of $CO_2$. Based on the equilibria, 
\begin{eqnarray*}
\left[H_2CO_3\right] &=& K_H \; p_{CO_2}   \\[3pt]
\left[HCO_3^-\right] &=& K_1 \; \frac{\left[H_2CO_3\right]}{\left[H+\right]} \\[3pt]
\left[CO_3^=\right] &=& K_2 \; \frac{\left[HCO_3-\right]}{\left[H+\right]}
\end{eqnarray*}
substitute those equilibria into the $C(IV)$ equation giving
\begin{eqnarray*}
H_{eff} &=& \left( K_H  p_{CO_2} + K_1  \frac{[H_2CO_3]}{[H+]} + K_2  \frac{[HCO_3^-]}{[H+]} \right)/p_{CO_2}   
\end{eqnarray*}
and further substitution gives
\begin{eqnarray*}
H_{eff} &=& K_H + K_H  \frac{K_1}{\left[H+\right]} + K_H\frac{K_1 K_2}{[H+]^2} 
\end{eqnarray*}
resulting in
\begin{eqnarray*}
H_{eff} &=& K_H  \left(1+  \frac{K_1}{[H+]}  \left(1 +   \frac{K_2}{[H+]}\right)\right) 
\end{eqnarray*}
\end{document}  